\documentclass[addpoints]{exam}

\usepackage{hyperref}

% Header and footer.
\pagestyle{headandfoot}
\runningheadrule
\runningfootrule
\runningheader{CS 440}{HW 4: Raytracer}{Fall 2022}
\runningfooter{}{Page \thepage\ of \numpages}{}
\firstpageheader{}{}{}

% \qformat{{\large\bf \thequestion. \thequestiontitle}\hfill[\totalpoints\ points]}
\qformat{{\large\bf \thequestion. \thequestiontitle}\hfill}
\boxedpoints

\title{Homework 4: Raytracer}
\author{CS 440 Computer Graphics\\Habib University}
\date{Fall 2022}

\begin{document}
\maketitle


In this assignment, you will implement an extendable ray tracing engine.

It is important to understand the setup. A \textit{sampler} uses a \textit{camera} to shoot \textit{ray}s through a \textit{viewplane}. Each ray traverses a scene which consists of \textit{geometry} whose appearance is determined through \emph{material}s. The ray returns some \textit{shading information} which is used to determine the shade to assign to a pixel in an \textit{image}.

The accompanying \texttt{raytacer} directory and its sub-directories contain classes modeling all of the above and more.
\begin{itemize}
\item The \texttt{geometry} folder contains declarations of an abstract class, \texttt{Geometry}, and concrete classes  \texttt{Plane}, \texttt{Sphere}, and \texttt{Triangle} which inherit from \texttt{Geometry}. You have to provide the required implementations.
\item The \texttt{materials} folder contains declarations of an abstract class, \texttt{Material}, and a concrete class  \texttt{Cosine} that inherits from \texttt{Material}. The cosine material assigns color based on the angle between the ray and the normal at the hit point. You have to provide the required implementations.
\item The \texttt{samplers} folder contains declarations of an abstract class, \texttt{Sampler}, and a concrete class  \texttt{Simple} that inherits from \texttt{Sampler}. The simple sampler samples each pixel once through its center. You have to provide the required implementations.
\item The \texttt{cameras} folder contains declarations of an abstract class, \texttt{Camera}, and concrete classes, \texttt{Parallel} and \texttt{Perspective}, that inherit from \texttt{Camera}. A parallel camera stores the direction of projection and a perspective camera stores the position of the center of projection. You have to provide the required implementations.
\end{itemize}

The \texttt{world} folder includes declarations of 2 classes. \texttt{ViewPlane} contains information on the view plane. \texttt{World} contains all the information required to render the scene--the geometry and associated materials, the view plane, the camera and sampler, and the background color. \texttt{World::build} populates the scene. You will reimplement this function each time to define a new image to be rendered. The \texttt{build} folder includes some sample implementations of \texttt{World::build}. You have to provide the required implementations.

The \texttt{utilities} folder includes declarations of utility classes, notably \texttt{Image} and \texttt{ShadeInfo}. \texttt{Image} holds pixel colors and writes an image to file in \href{https://en.wikipedia.org/wiki/Netpbm_format#PPM_example}{PPM format} (ASCII version). \texttt{ShadeInfo} contains all the information required for shading a point. These classes refer to classes from the \texttt{world} folder. Other useful classes in this folder are \texttt{Ray}, \texttt{Vector3D}, and \texttt{Point3D}. You have to provide the required implementations.

Everything comes together in \texttt{raytracer.cpp} in order to render a scene.

Go over the provided files to make sure that you understand the overall structure. Make sure to do so in a top-down manner. That is, start with \texttt{raytracer.cpp}, then look into the classes that it includes, then those that are included in these classes, and so on. At each stage, make sure to relate the content of the files with what you know about raytracing.

\section*{Task}
Your task is to implement the necessary classes such that running the  provided \texttt{raytracer.cpp} as-is renders the scene defined in \texttt{World::build}. You may use the 3 sample \texttt{build} functions provided in the \texttt{build} folder. Share your renders on \href{https://web.yammer.com/main/org/habib.edu.pk/groups/eyJfdHlwZSI6Ikdyb3VwIiwiaWQiOiIxMTc4MTkzMTAwODAifQ/all}{Yammer}.

\section*{Credits}

The code is adapted from that provided by \href{http://www.raytracegroundup.com/}{Kevin Suffern}.

\end{document}
%%% Local Variables:
%%% mode: latex
%%% TeX-master: t
%%% End:
